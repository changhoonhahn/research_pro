\clearpage

\setcounter{thefigs}{0}

\clearpage
\stepcounter{thefigs}
\begin{figure}
\figurenum{\fignum}
\plotone{spec_lrg.ps}
\caption{\label{spec_lrg} Best fit LRG spectral template in the rest
frame (normalization is for a 1 $M_\odot$ galaxy located 10 pc away,
or equivalently a $10^{12}$ $M_\odot$ galaxy located 10 Mpc away).}
\end{figure}

\clearpage
\stepcounter{thefigs}
\begin{figure}
\figurenum{\fignum}
\plotone{sfh_lrg.ps}
\caption{\label{sfh_lrg} Star-formation history corresponding to LRG
	spectral template of Figure \ref{spec_lrg}. Top panel shows the
	number of stars formed per logarithmic time interval ($t$ is
	expressed in years, curve is normalized for a $10^{12}$ $M_\odot$
	galaxy). Almost all of the stars are formed in the first couple of
	billion years --- note that the recent ``spike'' is represents a
	tiny fraction ($\sim 10^{-8}$) of the total number of stars. Bottom
	panel shows the mean metallicity of the population as a function of
	time. Note that the details of these functions are rather poorly
	constrained. }
\end{figure}

\clearpage
\stepcounter{thefigs}
\begin{figure}
\figurenum{\fignum}
\plotone{lrg_colors.ps}
\caption{\label{lrg_colors} SDSS colors of LRGs as a function of
	redshift. The greyscale is the conditional distribution of color
	within each redshift bin. The thin lines are the 10\%, 25\%, 50\%,
	75\%, and 90\% quantiles of the distribution. The thick line
	is the prediction of the model. The $u$ band is not included in the
	fit, and the $u$ magnitudes of most LRGs are poorly known. The other
	colors fit the models reasonably well. This model, remember, is
	given incredible freedom, meaning that the above agreement is the
	best one can do with the stellar population synthesis code of
	\citet{bruzual03a}. }
\end{figure}

\clearpage
\stepcounter{thefigs}
\begin{figure}
\figurenum{\fignum}
\plotone{sfh_templates.ps}
\caption{\label{sfh_templates} Similar to Figure \ref{sfh_lrg}, but
for the five global templates. Again, there are many degeneracies in
the fit parameters (though there are fewer in the actual spectra), so
these figures need to be interpreted appropriately.}
\end{figure}

\clearpage
\stepcounter{thefigs}
\begin{figure}
\figurenum{\fignum}
\plotone{fullfits.ps}
\caption{\label{fullfits} Color residuals (defined explicitly in the
	text) of GALEX, SDSS, and 2MASS observations relative to our best
	fit 5-template model.  The greyscale is the conditional distribution
	of the color residual given the redshift.  The thin lines are the
	10\%, 25\%, 50\%, 75\%, and 90\% quantiles of the distribution. The
	thick dashed lines show the estimated 1$\sigma$ uncertainties in the
	colors from the photometric catalogs. Relative to the uncertainties,
	there are no significant biases or redshift trends in these fits. }
\end{figure}

\clearpage
\stepcounter{thefigs}
\begin{figure}
\figurenum{\fignum}
\plotone{goods.ps}
\caption{\label{goods} Color residuals in fit using the standard five
templates to GOODS data, compared to the typical uncertainties (thick
dashed lines). Note that the fits always do poorly on the $H$ band,
which we believe to be a catalog error. }
\end{figure}

\clearpage
\stepcounter{thefigs}
\begin{figure}
\figurenum{\fignum}
\plotone{goods_special.ps}
\caption{\label{goods_special} Same as Figure \ref{goods}, but fitting
using five templates specially designed for GOODS. These templates
have smaller residuals in many respects but still fail to fit the $H$
band data. }
\end{figure}

\clearpage
\stepcounter{thefigs}
\begin{figure}
\figurenum{\fignum}
\plotone{specfit.ps}
\caption{\label{specfit} Best fit model spectra based on the five
template fit to $g$, $r$ and $i$ fluxes, compared to the original
SDSS spectra from which we computed those fluxes.  The models and the
original spectra agree very well.}
\end{figure}

\clearpage
\stepcounter{thefigs}
\begin{figure}
\figurenum{\fignum}
\plotone{twomass_resid.ps}
\caption{\label{twomass_resid} Similar to Figure \ref{fullfits} but
for galaxies observed in both SDSS and 2MASS and only using SDSS and
2MASS bands. The fits are to the SDSS and 2MASS data together.}
\end{figure}

\clearpage
\stepcounter{thefigs}
\begin{figure}
\figurenum{\fignum}
\plotone{twomass_predicted.ps}
\caption{\label{twomass_predicted} Similar to Figure
\ref{twomass_resid} but now the fits are {\it only} to the SDSS
bands. The residuals in the 2MASS bands remain very small, indicating
that the 2MASS measurements do not add a lot of information about
these galaxies.}
\end{figure}

\clearpage
\stepcounter{thefigs}
\begin{figure}
\figurenum{\fignum}
\plotone{galex_resid.ps}
\caption{\label{galex_resid} Similar to Figure \ref{twomass_resid} but
for galaxies observed in both SDSS and GALEX and only using SDSS and
GALEX bands. The fits are to the SDSS and GALEX data together.}
\end{figure}

\clearpage
\stepcounter{thefigs}
\begin{figure}
\figurenum{\fignum}
\plotone{galex_predicted.ps}
\caption{\label{galex_predicted} Similar to Figure
\ref{galex_resid} but now the fits are {\it only} to the SDSS
bands. }
\end{figure}

\clearpage
\stepcounter{thefigs}
\begin{figure}
\figurenum{\fignum}
\plotone{galex_kcorrect.ps}
\caption{\label{galex_kcorrect} $K$-corrections as a function of
redshift in the GALEX near (N) and far (F) UV bands. }
\end{figure}

\clearpage
\stepcounter{thefigs}
\begin{figure}
\figurenum{\fignum}
\plotone{twomass_kcorrect.ps}
\caption{\label{twomass_kcorrect} $K$-corrections as a function of
redshift in the 2MASS $J$, $H$ and $K_s$ bands.}
\end{figure}

\clearpage
\stepcounter{thefigs}
\begin{figure}
\figurenum{\fignum}
\plotone{sdss_kcorrect.ps}
\caption{\label{sdss_kcorrect} $K$-corrections as a function of
redshift in the SDSS $u$, $g$, $r$, $i$, and $z$ bands.}
\end{figure}

\clearpage
\stepcounter{thefigs}
\begin{figure}
\figurenum{\fignum}
\plotone{main_kcorrect.ps}
\caption{\label{main_kcorrect} Same as Figure \ref{sdss_kcorrect}, but
$K$-correcting to the \band{0.1}{u}, \band{0.1}{g}, \band{0.1}{r},
\band{0.1}{i}, and \band{0.1}{z} bands.}  
\end{figure}

\clearpage
\stepcounter{thefigs}
\begin{figure}
\figurenum{\fignum}
\plotone{mass_to_garching.ps}
\caption{\label{mass_to_garching} Our galaxy stellar mass estimates
	$M_\ast$ compared to those of \citet{kauffmann03a} $M_{s,\ast}$, as
	a function of stellar mass (top panel) and of color (bottom
	panel). The greyscale is the conditional distribution
	$M_{s,\ast}/M_\ast$ on each quantity . The lines are the 10\%, 25\%,
	50\%, 75\% and 90\% quantiles.}
\end{figure}

\clearpage
\stepcounter{thefigs}
\begin{figure}
\figurenum{\fignum}
\plotone{mtol.ps}
\caption{\label{mtol} Mass-to-light ratios of galaxies in the $V$ band
	(in solar units) as a function of galaxy $B-V$ color. The solid line
	satisfies the relationship $\log_{10}(M/L_V) = 1.40 (g-r) - 0.73$,
	given by \citet{bell01b} for their sample of spiral galaxies. 
	Their estimates and ours agree for $B-V < 0.8$, where spiral
	galaxies dominate the galaxy population.}
\end{figure}

\clearpage
\stepcounter{thefigs}
\begin{figure}
\figurenum{\fignum}
\plotone{umr_bg.ps}
\caption{\label{umr_bg} Fraction of the total star-formation that has
	occurred in the previous 300 Myr $b_{300}$, as a function of
	restframe $u-r$ color, for SDSS galaxies. The greyscale is the
	conditional distribution of $b_{300}$ on $u-r$. The lines are the 10\%,
	25\%, 50\%, 75\% and 90\% quantiles.  For comparison we show the
	results of using the scaling of \citet{hopkins03a}.}
\end{figure}

\clearpage
\stepcounter{thefigs}
\begin{figure}
\figurenum{\fignum}
\plotone{lemma.eps}
\caption{\label{lemma} Example diagram of definition of $\phi$ in the
	text. $\chi^2$ represents the actual $\chi^2$ while $\phi$ is
	designed such that it exceeds $\chi^2$ except at the single point
	$W$. By minimizing $\phi$ we can find a point $W^\ast$ that we know
	to have an equal or better $\chi^2$ than that at $W$. }
\end{figure}
